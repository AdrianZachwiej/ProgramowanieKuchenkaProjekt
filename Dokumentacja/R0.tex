% ********** Rozdział 0 **********
\chapter{Wymagania projektu}
\section{Wymagania Funkcjonalne}

\begin{enumerate}
    \item \textbf{Rozpoczęcie podgrzewania:} \\
     Użytkownik może wybrać produkt z listy dostępnych produktów, wprowadzić czas gotowania w sekundach.
    
    \item \textbf{Informacja z podgrzewania:} \\
    Użytkownik może przerwać podgrzewanie, po zakończeniu gotowania użytkownik otrzymuje informację o stanie gotowania (np. czy jedzenie jest spalone, niedogotowane lub gotowe).
    
    \item \textbf{Zarządzanie produktami:} \\
    Użytkownik może dodawać nowe produkty do listy dostępnych produktów wraz z czasem gotowania, usuwać istniejące produkty, edytować istniejące produkty.
    
    \item \textbf{Historia użycia:} \\
    Mikrofalówka rejestruje historię każdego użycia, wraz z informacjami o produkcie, czasie gotowania i wyniku (np. czy jedzenie było gotowe, niedogotowane lub spalone), użytkownik może przeglądać historię użycia.
    
    \item \textbf{Czyszczenie mikrofalówki:} \\
    Po każdych pięciu użyciach mikrofalówka wymaga czyszczenia, użytkownik może zresetować licznik użycia po wykonaniu czyszczenia.
    
    \item \textbf{Interfejs użytkownika:} \\
    Użytkownik powinien mieć prosty interfejs do nawigacji po funkcjach mikrofalówki. Po zakończeniu operacji użytkownik powinien mieć możliwość powrotu do głównego menu.
\end{enumerate}

\newpage
\section{Wymagania Niefunkcjonalne}

\begin{enumerate}
    \item \textbf{Wydajność:} \\
    Mikrofalówka powinna być responsywna i szybka w działaniu, zapewniając użytkownikowi płynne doświadczenie użytkowania.
    
    \item \textbf{Diagnostyka:} \\
    Mikrofalówka powinna być poddana testom jednostkowym, aby zapewnić wysoką jakość i niezawodność działania.
    
    \item \textbf{Skalowalność:} \\
    Projekt mikrofalówki powinien być zaprojektowany w taki sposób, aby łatwo można było dodawać nowe funkcje i rozszerzać jego możliwości w przyszłości.
    
    \item \textbf{Skalowalność:} \\
    System powinien być łatwo skalowalny, umożliwiając dostosowanie się do wzrostu liczby klientów i zwiększenia obciążenia systemu. Architektura systemu powinna być elastyczna i umożliwiać dodawanie nowych zasobów w miarę potrzeb.
    
    \item \textbf{Dostępność:} \\
    Aplikacja mikrofalówki powinna być dostępna dla osób z różnymi poziomami umiejętności.
    
\end{enumerate}



% ********** Koniec rozdziału **********
