% ********** Rozdział 2 **********
\chapter{Opis techniczny projektu}
\section{Wykorzystywane technologie}

Projekt został zaimplementowany w języku C\# z wykorzystaniem platformy .NET. Do zarządzania bazą danych używa plików tekstowych. Projekt wykorzystuje podejście programowania obiektowego.

\subsection{Narzędzia}
Do implementacji projektu użyto następujących narzędzi:
\begin{itemize}
    \item Microsoft Visual Studio - środowisko programistyczne do tworzenia aplikacji w języku C\#.
    \item LaTeX - do tworzenia dokumentacji w formacie PDF zgodnej z wymaganiami.
\end{itemize}
\subsection{Minimalne wymagania sprzętowe}
Minimalne wymagania sprzętowe dla uruchomienia projektu są następujące:
\begin{itemize}
    \item Procesor: Procesor zgodny z architekturą x86 lub x64 o częstotliwości 1 GHz lub więcej.
    \item Minimum 512 MB RAM.
    \item Minimum 100 MB wolnego miejsca na dysku twardym.
    \item  Windows 7 lub nowszy, lub dowolny system obsługujący środowisko uruchomieniowe .NET Framework 4.7.2 lub nowsze.
\end{itemize}
\subsection{Zarządzanie danymi oraz baza danych}

Do zarządzania danymi w projekcie wykorzystano klase ProductManager, zaimplementowano metody obsługujące operacje CRUD (Create, Read, Update, Delete) na tekstowej bazie danych. Klasa Validator sprawdza poprawność wprowadzonych danych.

% ********** Koniec rozdziału **********
